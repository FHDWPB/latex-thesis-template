%!TEX root = ../Thesis.tex
\section{Installation}
\label{instal}

\subsection{TeX-Distribution}

Für die Arbeit mit \LaTeX ist eine TeX-Distribution erforderlich. 

\subsubsection{Windows}

Unter Windows ist MiKTeX die Standard-{\LaTeX}-Distribution. Der MikTex-Installer kann unter \url{http://miktex.org/download} heruntergeladen werden.

\subsubsection{Linux}

Die Standard-{\LaTeX}-Distribution unter Linux ist Tex Live, welche über die gängigen Software-Repositories installiert werden kann.

Unter Debian/Ubuntu kann die Installation der erforderlichen Pakete mittels der folgenden Befehlen durchgeführt werden:

\begin{quoting}
\texttt{sudo apt-get install texlive-latex-base}\\
\texttt{sudo apt-get install texlive-latex-recommended}\\
\texttt{sudo apt-get install texlive-fonts-recommended}\\
\texttt{sudo apt-get install biblatex}\\
\texttt{sudo apt-get install biber}
\end{quoting}

\subsection{PDF-Viewer}

\subsubsection{Windows}

Als PDF-Viewer unter Windows bietet sich der freie Sumatra PDF Viewer an: \url{http://blog.kowalczyk.info/software/sumatrapdf/download-free-pdf-viewer-de.html}

\subsubsection{Linux}

Die installierten Standard-PDF-Viewer unter Linux können problemlos genutzt werden.

\subsection{Literaturverwaltung}

Für die Verwaltung von Quellen eignet sich das freie, Cloud-basierte Mendely: \url{http://www.mendeley.com/download-mendeley-desktop/}. 

\begin{figure}[hbt]
\centering
\begin{minipage}[t]{1\textwidth} % Breite, z.B. 1\textwidth		
\caption{Mendeley Referenzmanager} % Überschrift
\includegraphics[width=1\textwidth]{img/Mendeley-destop-screenshot}\\ % Pfad
\source{\url{http://dominique-fleury.com/?p=302}} % Quelle
\end{minipage}
\end{figure}

\subsection{Texteditor}

Als Texteditor für \LaTeX wird Sublime Text (\url{http://www.sublimetext.com}) empfohlen. Zur Arbeit mit Latex ist das Plugin \emph{LaTeXTools} erforderlich (\url{https://github.com/SublimeText/LaTeXTools}).

\begin{figure}[hbt]
\centering
\begin{minipage}[t]{1\textwidth} % Breite, z.B. 1\textwidth		
\caption{Sublime Texteditor} % Überschrift
\includegraphics[width=1\textwidth]{img/sublime.png}\\ % Pfad
\source{\url{http://www.sublimetext.com/screenshots/alpha_goto_anything2_large.png}} % Quelle
\end{minipage}
\end{figure}